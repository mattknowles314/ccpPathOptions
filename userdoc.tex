\documentclass[11pt]{amsart}
\usepackage[utf8]{inputenc}
\usepackage[a4paper, total={6in,8in}, portrait, margin=1in]{geometry}
\usepackage{amssymb}

%Add any packages you need here
\usepackage{graphicx}
\usepackage{amsmath}
\usepackage{amsfonts}
\usepackage{float}
\usepackage{listings}
\usepackage[misc]{ifsym}
\usepackage{indentfirst} 
\usepackage{amsthm}
\usepackage{appendix}


%Any functions you wanna define, pop 'em here
\newtheorem{theorem}{Theorem}[section]
\newtheorem{remark}[theorem]{Remark}
\newtheorem{definition}[theorem]{Definition}
\newtheorem{example}[theorem]{Example}
\newtheorem{lemma}[theorem]{Lemma}

\title{Automated Stock Trading using Neural Networks}
\author{Matthew Knowles}
\date{Autumn Term 2021}

\begin{document}

\maketitle

\section{DisplayDoubleVector}
The purpose of this function is to display a double vector to the console. To use this function, simply pass the 2D vector (of any type)
to the function, and it will be outputted to the console. Since this is a template function, a 2D vector of any type may be outputted without 
issue. 

\section{ReadCSVFile}
This class is used for reading in a CSV file of stock data. To create an intial object of this type, a path to the file must be passed upon 
creation. I.e \textit{ReadCSV csv("Data/StockDataMar2022.csv");}. The data from the CSV file pointed to is the converted to a double vector 
by calling the member function \textit{CSV2VEC()}. 


\end{document}